\section{Introduction}
A music genre is a conventional label on the musical piece which characterizes it as having certain features, conventions, or characteristics. It is quite a complicated problem to say precisely how genres are distinguished. The genre often dictates the style and rhythm of the audio of the song. It seems much harder to define the music genre by lyrics alone, even from a human perspective. Therefore, it is quite an interesting topic to try making such a distinction based on song text. Similar research has already been conducted, but this topic is yet to be fully explored.

The song's lyrics are often related to its melody and rhythm. It is also common for different genres to raise different topics. It was already shown that a combination of audio and text features gets better results than using only audio features \cite{mayer2011Ref}. Furthermore, lyrics may be more accessible and easier to process than audio. Therefore, lyrics classification seems to be an interesting field of study both for its own and for its potential connection with audio features.

We have previously explored different methods for lyrics-based genre classification. Our study included testing different methods of obtaining text embeddings, such as GloVe, word2vec, BERT, and varying classification models, such as Naive Bayes, Linear Support Vector Machine, XGBoost, and Convolutional Neural Network. In this research, we focus on improving obtained results. We add sentiment analysis for the classifier to see the lyrics from a different perspective. We build a model containing two separate models, where the first one decides whether the song belongs to the \textit{Rock} genre or not and the second one classifies samples into remaining genres. 

We also create a small dataset to explore using available APIs. Then we test some of the models on it.

The research paper is divided into multiple sections. In section \ref{related_work} we describe related works in the domain of MGC. Section \ref{approach} presents used datasets and the preprocessing done. Furthermore, used methods and models are characterized. In section \ref{hyperparameters} we show the hyperparameters of the models used. In section \ref{experiments} we demonstrate conducted experiments and obtained results. The whole project is concluded in section \ref{conclusion}.
