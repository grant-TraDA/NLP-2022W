%
% File main.tex
%
% Contact: car@ir.hit.edu.cn, gdzhou@suda.edu.cn
%%e.agirre@ehu.es or Sergi.Balari@uab.es
%% and that of ACL 08 by Joakim Nivre and Noah Smith

\documentclass[11pt]{article}
\usepackage{comment}
\usepackage{acl2015}
\setlength\titlebox{6cm}
\usepackage{times}
\usepackage{url}
\usepackage{hyperref}
\usepackage{latexsym}
\usepackage{biblatex}
\addbibresource{bibliography.bib}
\usepackage{csquotes}
\usepackage{tabularx}
\usepackage{xcolor}
\usepackage{svg}

\title{Music genre classification based on song lyrics - comparison between different word embedding techniques and classifiers \\Project Proposal for NLP Course, Winter 2022}

\author{Bartłomiej Eljasiak \\
  Warsaw University of Technology \\
  {\tt\small bartlomiej.eljasiak.stud@pw.edu.pl} \\\And
  Aleksandra Nawrocka \\
  Warsaw University of Technology \\
  {\tt\small aleksandra.nawrocka.stud@pw.edu.pl} \\
  \AND
  Dominika Umiastowska \\
  Warsaw University of Technology \\
  {\tt\small dominika.umiastowska.stud@pw.edu.pl} \\\And 
  supervisor: Anna Wróblewska\\
  Warsaw University of Technology \\
  {\tt\small anna.wroblewska1@pw.edu.pl}\\}

\date{}

\begin{document}
\raggedbottom

\maketitle

\vspace{5em}

\begin{abstract}

Music genre classification (MGC), although a well-known task, still remains challenging in the domain of Music Information Retrieval. We tackle the problem of MGC based solely on lyrics and try to solve it using a solution composed of a state-of-the-art word embedding method and a separate classification model. Our main contribution is the comparison between different word embedding methods and classification techniques, which in the domain of MGC is currently lacking. The novelty comes with an additional approach in the form of testing the impact of enriching the lyrics with the title of the song.

\end{abstract}



\section{Introduction}
A music genre is a conventional label on the musical piece which characterizes it as having certain features, conventions, or characteristics. It is quite a complicated problem to say precisely how genres are distinguished. The genre often dictates the style and rhythm of the audio of the song. It seems much harder to define the music genre by lyrics alone, even from a human perspective. Therefore, it is quite an interesting topic to try making such a distinction based on song text. Similar research has already been conducted, but this topic is yet to be fully explored.

The song's lyrics are often related to its melody and rhythm. It is also common for different genres to raise different topics. It was already shown that a combination of audio and text features gets better results than using only audio features \cite{mayer2011Ref}. Furthermore, lyrics may be more accessible and easier to process than audio. Therefore, lyrics classification seems to be an interesting field of study both for its own and for its potential connection with audio features.

We have previously explored different methods for lyrics-based genre classification. Our study included testing different methods of obtaining text embeddings, such as GloVe, word2vec, BERT, and varying classification models, such as Naive Bayes, Linear Support Vector Machine, XGBoost, and Convolutional Neural Network. In this research, we focus on improving obtained results. We add sentiment analysis for the classifier to see the lyrics from a different perspective. We build a model containing two separate models, where the first one decides whether the song belongs to the \textit{Rock} genre or not and the second one classifies samples into remaining genres. 

We also create a small dataset to explore using available APIs. Then we test some of the models on it.

The research paper is divided into multiple sections. In section \ref{related_work} we describe related works in the domain of MGC. Section \ref{approach} presents used datasets and the preprocessing done. Furthermore, used methods and models are characterized. In section \ref{hyperparameters} we show the hyperparameters of the models used. In section \ref{experiments} we demonstrate conducted experiments and obtained results. The whole project is concluded in section \ref{conclusion}.


\section{Related works}\label{related_work}

Music genre classification (MGC) is at this point a well-known research problem and a subdomain of Music Information Retrieval (MIR). Culture and therefore music avoids strict barriers and definitions, nevertheless, each piece of music is usually categorized into one or more genres. MGC enables us to study this categorization, explore similarities and differences between various genres or even construct a taxonomy. 

In the past, due to heavy computational limitations, the main focus of MGC was put on finding the best features for classification purposes. In \cite{oldFeatures} such features were e.g. \textit{AverageSyllablesPerWord} or \textit{SentenceLengthAverage}. Naturally, word embedding played an important role in extracting information from lyrics and the use of simple methods like \textit{bag-of-words} can be found in various papers \cite{mgc_example_1, liang2011music}. With time, an increasing amount of focus was put strictly on embeddings themselves, developing novel and improved representations. 

Currently, all state-of-the-art approaches for MGC utilizing lyrics rely heavily on word embeddings. In a recent publication \cite{musicWordEmbed} an attempt was made to train the embedding model strictly on lyrics. Unfortunately, the significance of the work is hard to assess due to the lack of usage of this model.

It is also rather common to approach MGC in a multi-modal manner. Usage of the audio itself has to be second if not the most popular source of information with many published articles \cite{audio_1dcnn, audio_attention, audio_reviews_cover, oldFeatures, oldAudio}. Other less trivial data sources are symbolic \cite{symbolic}, culture \cite{oldFeatures}, text reviews \cite{audio_reviews_cover}, and cover art \cite{audio_reviews_cover}. One could say that at this stage researchers experiment with enriching the pieces of music with any meaningful data possible.

The datasets that we work on are:
\begin{itemize}
    \item \textit{Song lyrics from 79 musical genres} dataset from Kaggle website \cite{KaggleDataset},
    \item \textit{MetroLyrics} dataset processed and put in a GitHub repository \cite{GithubDataset},
    \item our own dataset created using Spotify API \cite{Spotify} and Genius API \cite{Genius}.
\end{itemize}



\section{Approach and research methodology}\label{approach}

\subsection{Datasets}\label{datasets}
We have found two datasets which we decided to work on:
\begin{itemize}
    \item \textit{Song lyrics from 79 musical genres} dataset from Kaggle website \cite{KaggleDataset},
    \item \textit{MetroLyrics} dataset processed and put in a GitHub repository \cite{GithubDataset}.
\end{itemize}

In the description of the first dataset, we can find the information that the dataset consists of 379 893 song lyrics from 4239 artists. Around 50\% of the song lyrics are in English and we test our models on them. Information about the artists is kept in a separate file and contains a list of music genres each artist is connected with. As we predict only one music genre for each song we preprocess this dataset by reducing these lists to individual genres (we take the first one from the list) and assigning them to song lyrics of appropriate artists. Furthermore, we preprocess song lyrics as they contain punctuation and span across multiple lines.

By contrast, the second dataset required minimal work on our site. It was initially published on Kaggle website and consisted of 362 237 song lyrics from 18231 artists. The majority of song lyrics (probably around 60\%) were in English. Unfortunately, this dataset was removed from Kaggle website and we were not able to find it in its original form anywhere else. We have found a preprocessed version of it in a GitHub repository of a students' project performed by University of California students in 2018. This version's song lyrics have punctuation removed and contain only one genre for each entry.

\subsection{Datasets preparation}\label{Datasets preparation}

First of all, we conducted some basic preparations of datasets.
For \textit{MetroLyrics} we decided to remove genre \textit{Other} and merge genres \textit{Country} and \textit{Folk} since  our research showed they are very similar, and additionally the second was significantly smaller in samples. As for \textit{Song lyrics from 79 musical genres}, we filtered songs to only English ones and omitted the genre \textit{Pop/Rock} since it is ambiguous.

After conducting a few simple tests on the whole dataset we decided to limit ourselves to a much smaller part of the observations. There were two main reasons for that: limited resources and time - we would not manage to conduct all desired tests on the whole dataset, and second - the dataset was very strongly unbalanced and therefore accuracy was often significantly bigger than balanced accuracy. To solve both problems we decided to limit ourselves to only five genres with the biggest number of samples: \textit{Rock}, \textit{Pop}, \textit{Metal}, \textit{Hip-hop} and \textit{Country}.

In the end, we conducted tests on two datasets:
\begin{itemize}
    \item Balanced dataset with lyrics from 5 genres, $116,120$ observations in total, created from both \textit{MetroLyrics} and \textit{Song lyrics from 79 musical genres} datasets.

    \item Unbalanced dataset with lyrics from 5 genres, $262,122$ observations in total, created from both \textit{MetroLyrics} and \textit{Song lyrics from 79 musical genres} datasets.
\end{itemize}

\subsection{Text preprocessing}

Since in our project, we use embeddings that may be or even should be used on the raw text we decided to test two approaches: one with strong and the second with weak preprocessing of text.

Weak preprocessing consists of deleting numbers in the text and some words characteristic of lyrics notation (e.g. "VERSE", "CHORUS", "2x"). We decided to also expand contractions since they were proven to be troublesome in further preprocessing such as tokenization or removing stopwords. We deleted all special characters since interpunction was used inconsistently in most songs (e.g. lyrics usually didn't have a division for sentences). We also lowered the whole text since every verse began with a capital letter which usually had nothing to do with starting a sentence. Thanks to that we could also use uncased versions of embeddings.

Strong preprocessing consists of, besides the above, tokenization, lemmatization, and deleting stop words. For these steps, we used \textit{nltk} package.



\subsection{Division into two models}

Since one of the datasets is very unbalanced (the amount of Rock genre observations is two times more numerous than all other genres together) we decided to try an approach that will classify the largest genre class separately from the rest. Therefore we prepared a 2-step CNN classifier that consists of two sub-classifiers: a BinaryCNN classifier which only recognizes if lyrics are of the one selected genre or not, and a multiclass CNN classifier (the same is used as a normal classifier for all genres) which classifies all other genres.

\begin{figure}[!h]
\centering
\includegraphics[width=7cm]{plots/2step_cnn_schema.drawio.png}
\caption{2-step CNN architecture}
\end{figure}

In this way, we hope to better recognize and separate the Rock genre and influence significantly balanced accuracy.


\subsection{Creation of the dataset}
We decided to create our own dataset using Spotify API \cite{Spotify} and Genius API \cite{Genius}. We wanted to use Spotify to get tracks from chosen genres and Genius to get lyrics of returned songs. At first, we utilized the \textit{Get Recommendations} endpoint. Unfortunately, the recommended songs repeated a lot between different API calls. That meant that after dropping duplicates we were left with a few times less observations than desired. To deal with this issue, we decided to use the \textit{Search for Item} endpoint. This endpoint allows specifying genre but limits the possible output to the first 1000 tracks. Another problem that we faced was that Genius did not always have the lyrics to the songs provided by Spotify. And even if it did sometimes instead of proper lyrics text of books or different lists of artists and the titles of their songs were returned. Because of that the created dataset needed also manual data cleaning. At last, in that way, we created the dataset containing 5 genres and 4,092 observations. Table \ref{tab:our_dataset_genres} presents the number of observations in each genre.

\begin{table}[h]
\centering
\begin{tabular}{l|r}
\textbf{Genre} & \textbf{Number of observations} \\\hline
country & 896 \\
metal & 887 \\
pop & 815 \\
rock & 767 \\
hip-hop & 727 \\
\end{tabular}
\caption{The number of observations in each genre in created dataset}
\label{tab:our_dataset_genres}
\end{table}


\section{Experiments and results}\label{experiments}

\subsection{New models and embeddings}
 In this project, several embedding models were tested, but it was soon rather clear that the best results will be obtained by the most complex embedding models. Additionally, we wanted to decrease the input size of classifiers to reduce the training time and complexity of several models. Described goals were achieved by introducing new embedding models with aggregated word embeddings. Said models are \texttt{DistilBERT} and \texttt{SentenceTransformerMPNET} with word embeddings transformed to singular sentence embedding, by aggregating them. This seemingly minor change greatly increased accuracy scores across all classifiers and set a new standard for future experiments.

 We also used two new classifiers: DenseNet which consists of a few fully connected dense layers, and the 2-step CNN closer described before. 
 
 Results of our experiments on a balanced dataset are visible in \ref{tab:distilbert_results} and \ref{tab:mpnet_results} tables. $\Delta$ Acc. signify the difference between results acquired in this project for this dataset with the new embedding and the best results obtained in the previous project.

\begin{table}[!h]
\centering
\begin{tabular}{l|l|l|l}
\textbf{Classifier} & \textbf{Accuracy} & \textbf{F1-score} & \textbf{ $\Delta$ Acc.}  \\ \hline
NB         & 54.40\%           & 53.16\%           & + 11.16               \\
SVM          & 62.06\%           & 61.07\%           & + 11.45               \\
XGBoost             & 63.93\%           & 63.78\%           & \textbf{+ 18.46}      \\
\textbf{CNN}        & \textbf{65.47\%}  & \textbf{65.53\%}  & + 8.99                \\
DenseNet            & 64.96\%           & 64.99\%           & \multicolumn{1}{c}{-} \\
2StepCNN            & 62.21\%           & 57.97\%           & \multicolumn{1}{c}{-}
\end{tabular}
\caption{Results for the DistilBERT  on balanced dataset}
\label{tab:distilbert_results}
\end{table}

\begin{table}[!h]
\centering
\begin{tabular}{l|l|l|l}
\textbf{Classifier} & \textbf{Accuracy} & \textbf{F1-score} & \textbf{$\Delta$ Acc.}  \\ \hline
NB         & 56.23\%           & 55.97\%           & + 12.99               \\
SVM          & 62.83\%           & 61.66\%           & + 12.22               \\
XGBoost             & 62.45\%           & 62.31\%           & \textbf{+ 16.98}      \\
CNN                 & 65.06\%           & 64.30\%           & + 8.58                \\
\textbf{DenseNet}   & \textbf{66.42\%}  & \textbf{66.08\%}  & \multicolumn{1}{c}{-} \\
2StepCNN            & 63.34\%           & 60.44\%           & \multicolumn{1}{c}{-}
\end{tabular}
\caption{Results for the SentenceTransformerMPNET on balanced dataset}
\label{tab:mpnet_results}
\end{table}

As we can see, all results are significantly better than in the previous project which shows how important embedding is in text classification. Also, the new classifier DenseNet gives very good results, even given its quite simple structure. The comparison of the results is shown in Figure \ref{fig:embed_comparison}.

\subsection{Fine-tunning embedding models}
The natural next step was to improve the information contained in the embedding model by combining it with the classification layer and fine-tuning the combined architecture. Since a DistilBERT was pre-trained on a broad corpus and a wide range of texts, we hoped that by partially training it on lyrics it will be able to improve the quality of the created embeddings.  This task was realized on \texttt{small\_balanced} version of the available dataset. Unfortunately, repeated experiments did not bring the expected results and the model performance was not significantly improved. A comparison of both fine-tuned  and pre-trained models is presented in Table \ref{tab:tuning_results}. Lack of improvement may indicate problems in the training process but repeated experiments and tests on different models indicate that this was not the case. 


\begin{table}[h]
\centering
\begin{tabular}{l|r|r|r}
 \textbf{Embedding} & \textbf{Classifier} & \textbf{Accuracy} & \textbf{F1 score} \\\hline
pretrained & Dense & 0.6455 & 0.6431 \\
fine-tuned & Dense & \textbf{0.65} & \textbf{0.6682} \\
\end{tabular}
\caption{Results after fine-tuning}
\label{tab:tuning_results}
\end{table}

\subsection{Addition of sentiment analysis}
One of the main things that we wanted to test, was the impact of additional information added to the embeddings. At the same time since one of the assumptions of the project was working with only the lyrics, features that we could add to the embedding of the lyrics are all lyrics based. Our idea was to incorporate sentiment obtained by a separate model. Unfortunately, since we focused on using as hardware demanding embedding model there were little to no resources left for this sentiment model. After minor research we settled for \href{https://huggingface.co/gokuls/BERT-tiny-emotion-intent}{\texttt{BERT-tiny-emotion-intent}} model. It offers sentiment prediction in form of a six-dimensional vector measuring the sentiment as happy, sad, neutral, angry, excited, or frustrated. This is to some extent a redundancy since we expect the sentiment to be present in embeddings, at least to some degree, but this information may not be easy for the model to extract especially since our classification models are relatively simple.

We tested the influence of sentiment in two experiments. The first one was oriented on the changes in the performance of classifiers. As it can be seen in \ref{tab:sentiment_results} results for feature vector enriched with embeddings did not really change, when compared to \ref{tab:distilbert_results}. The only exception here is the XGBoost model which improved significantly, but this may be partially explained by the instability of the XGBoost which was observed. 

\begin{table}[h]
\centering
\begin{tabular}{l|r|r|r}
 \textbf{Classifier} & \textbf{Accuracy} & \textbf{Bal. Acc.} & \textbf{F1 score} \\\hline
Naive Bayes & 0.5272 & 0.5272 & 0.5095 \\
Linear SVM & 0.6054 & 0.6054 & 0.5998 \\
XGBoost & 0.6300 & 0.6300 & 0.6279 \\
CNN & 0.6433 & 0.6433 & \textbf{0.6433} \\
Dense & \textbf{0.6449} & \textbf{0.6449} & 0.6341 \\
\end{tabular}
\caption{Classification results for better embeddings }
\label{tab:sentiment_results}
\end{table}

The second experiment concerning sentiment included fine-tuning the combination of the embedding model, sentiment model and dense classifier. Same as with the previous first fine-tuning the results did not really change \ref{tab:sentiment_tuning_results}, and actually the model without sentiment achieves the best score, but only by a small margin of less than 1\%. 

\begin{table}[h]
\centering
\begin{tabular}{l|r|r}
 \textbf{sentiment}& \textbf{Accuracy} & \textbf{F1 score} \\\hline
yes &  0.6412 & 0.6638 \\
no &  \textbf{0.65} & \textbf{0.6682} \\
\end{tabular}
\caption{Results after fine-tuning with or without sentiment}
\label{tab:sentiment_tuning_results}
\end{table}

\subsection{Division into two models}

There were already shown some results for the 2-step CNN classifier on the balanced dataset in the previous section in tables \ref{tab:distilbert_results} and \ref{tab:mpnet_results}. In general, achieved by 2-step CNN accuracies were slightly worse than those of normal CNN and DenseNet.

But this classifier was created with the thought of an unbalanced dataset and therefore results on this dataset are the most important to us.  Therefore, we tested 2-step CNN and two other classifiers which gave the best results in previous tests on the unbalanced dataset (which is also much bigger than balanced) to see if separate recognition of the biggest class gives better results. Results are in Table \ref{tab:unbalanced_results_2_step} and the comparison is shown in Figure \ref{fig:metric_comparison}.

\begin{table}[!h]
\centering
\begin{tabular}{l|l|l|l}
\textbf{CLassifier} & \textbf{Accuracy} & \textbf{Bal. acc.} & \textbf{F1 score} \\ \hline
CNN                 & 64.74\%           & 50.71\%                & 60.88\%           \\
\textbf{NN}         & \textbf{65.23\%}  & 53.08\%                & \textbf{62.14\%}  \\
\textbf{2-step CNN} & 52.94\%           & \textbf{59.88\%}       & \textbf{54.11\%} 
\end{tabular}
\caption{Results for 2-step CNN, DenseNet, and CNN for DistilBERT on unbalanced dataset}
\label{tab:unbalanced_results_2_step}
\end{table}

Even though neither accuracy nor F1-score was better for 2-Step CNN, it proved to give higher balanced accuracy results. Therefore it may be better than other classifiers depending on the task at hand.

\subsection{Creation of the dataset}
We conducted multiple tests for our dataset as it was a lot smaller than the other ones. The results are presented in Tables \ref{tab:dataset_res}, \ref{tab:dataset_res_norm} and Figures \ref{fig:acc_dataset}, \ref{fig:f1_dataset}, \ref{fig:acc_dataset_norm}, \ref{fig:f1_dataset_norm}. The best embedding turned out to be DistilBERT. But when it comes to classifiers the results were quite similar so it makes more sense to use the most basic classifier to limit the computations.

\onecolumn



\begin{figure}
\centering
\includegraphics[width=0.8\linewidth]{plots/accuracy_small_balanced.pdf}
\caption{Comparison of DistilBERT and SentenceTransformerMPNET on balanced dataset}
\label{fig:embed_comparison}
\end{figure}


\begin{figure}
\centering
\includegraphics[width=0.8\linewidth]{plots/metrics_comparison_small.pdf}
\caption{Comparison of different metrics for different classifiers on unbalanced dataset}
\label{fig:metric_comparison}
\end{figure}

\begin{table}[h]
\centering
\begin{tabular}{l|l|r|r}
\textbf{Embedding} & \textbf{Classifier} & \textbf{Accuracy} & \textbf{F1 score} \\\hline
Smaller BERT & Naive Bayes & 0.379479 & 0.328668 \\
Smaller BERT & Linear SVM & 0.399837 & 0.353035 \\
Smaller BERT & XGBoost & 0.375407 & 0.374183 \\
Smaller BERT & CNN & 0.415309 & 0.401595 \\
Glove & Naive Bayes & 0.333876 & 0.259667 \\
Glove & Linear SVM & 0.320033 & 0.295821 \\
Glove & XGBoost & 0.343648 & 0.336383 \\
Glove & CNN & 0.374593 & 0.382742 \\
\textbf{DistilBERT} & \textbf{Naive Bayes} & \textbf{0.547231} & \textbf{0.550679} \\
DistilBERT & Linear SVM & 0.477199 & 0.373867 \\
DistilBERT & XGBoost & 0.518730 & 0.519839 \\
DistilBERT & CNN & 0.535016 & 0.530048 \\
DistilBERT & DenseNet & 0.516287 & 0.516995 \\
SentenceTransformerMPNET & CNN & 0.495928 & 0.499407 \\
SentenceTransformerMPNET & DenseNet & 0.517915 & 0.517484 \\
\end{tabular}
\caption{Results for the created dataset and unnormalized lyrics}
\label{tab:dataset_res}
\end{table}

\begin{figure}
\centering
\includegraphics[width=0.8\linewidth]{plots/accuracy_dataset.pdf}
\caption{Accuracy for the created dataset and unnormalized lyrics}
\label{fig:acc_dataset}
\end{figure}

\begin{figure}
\centering
\includegraphics[width=0.8\linewidth]{plots/f1_dataset.pdf}
\caption{F1-score for the created dataset and unnormalized lyrics}
\label{fig:f1_dataset}
\end{figure}

\begin{table}[h]
\centering
\begin{tabular}{l|l|r|r}
\textbf{Embedding} & \textbf{Classifier} & \textbf{Accuracy} & \textbf{F1 score} \\\hline
Smaller BERT & Naive Bayes & 0.310261 & 0.247440 \\
Smaller BERT & Linear SVM & 0.368078 & 0.352763 \\
Smaller BERT & XGBoost & 0.400651 & 0.395027 \\
Smaller BERT & CNN & 0.439739 & 0.439537 \\
Glove & Naive Bayes & 0.328176 & 0.236727 \\
Glove & Linear SVM & 0.383550 & 0.376734 \\
Glove & XGBoost & 0.372964 & 0.369424 \\
Glove & CNN & 0.389251 & 0.395433 \\
DistilBERT & Naive Bayes & 0.515472 & 0.495294 \\
DistilBERT & Linear SVM & \textbf{0.541531} & 0.492267 \\
DistilBERT & XGBoost & 0.503257 & 0.500318 \\
DistilBERT & CNN & 0.494300 & 0.494112 \\
DistilBERT & DenseNet & 0.518730 & \textbf{0.517378} \\
SentenceTransformerMPNET & CNN & 0.514658 & 0.509839 \\
SentenceTransformerMPNET & DenseNet & 0.500000 & 0.501151 \\
\end{tabular}
\caption{Results for the created dataset and normalized lyrics}
\label{tab:dataset_res_norm}
\end{table}

\begin{figure}
\centering
\includegraphics[width=0.8\linewidth]{plots/accuracy_dataset_norm.pdf}
\caption{Accuracy for the created dataset and normalized lyrics}
\label{fig:acc_dataset_norm}
\end{figure}

\begin{figure}
\centering
\includegraphics[width=0.8\linewidth]{plots/f1_dataset_norm.pdf}
\caption{F1-score for the created dataset and normalized lyrics}
\label{fig:f1_dataset_norm}
\end{figure}
\twocolumn

\section{Discussion}\label{discussion}
There are several contributions to this work. First of all, we were not able to find a detailed comparison of statistical and deep learning models. Often works focus on a selected group of models like transformers in case of \cite{sig_bert}, 1D CNNs in \cite{audio_1dcnn} or Support Vector Machines explored in \cite{oldAudio}.

What is more, often different datasets are being used, which overall makes the results difficult to compare. Our work compares older and newer models on different datasets. This creates a common ground and context for other works and researchers. When it comes to the comparison of our results to the current state-of-the-art, again the situation is a bit complex. State-of-the-art solution for different, in our opinion simpler, task of genre classification based on audio stands at 80.93\% achieved in \cite{audio_1dcnn}. The best lyrics-based prediction accuracy that we were able to find was 77.63\%, achieved by the BERT model in \cite{sig_bert}.

This cast a strong shadow on our best result of 57.17\% but again, those results are difficult to compare. Why? This is due to the fact that the authors of mentioned SOTA preprocessed the data in an undocumented way, significantly simplifying the original problem present in the dataset. Naturally, as mentioned numerously in this document we also made changes to the formulation of the original problem, but due to the missing information from other works we are not able to reproduce this process in a similar way, hence making the results non-conclusive. What is worse, the mentioned article makes use of a different dataset, which makes the accuracy even harder to compare. 



\section{Conclusion}\label{conclusion}

This work explored various approaches to music genre classification based solely on lyrics and was oriented on comparing numerous techniques. This was done in order to create a context for future researchers, which is currently missing in this area of study. In performed experiments, we used two different datasets with two types of inputs for models, four embedding techniques and four vastly different machine learning classifiers.

The final results are the following, 56.48\% accuracy on lyrics with BERT embeddings and CNN classifier on our balanced dataset, and 57.17\% accuracy on lyrics with titles for BERT embeddings and CNN classifier on our unbalanced dataset. Both of the datasets were described in detail in \ref{Datasets preparation}. Those results may seem low, but the reader ought to remember that in the case of this task flawless classification may not be possible and currently the best result of 77.63\% accuracy, whereas higher, comes with a set of problems described in \ref{discussion}.

\section{Future work}\label{future_work}
The results of classification could, possibly to a large extent, still be improved and the final methods presented in this document ought to be treated as a decent starting point for future researchers. 

Based on all the knowledge gained while working on this project we humbly want to present several ideas, which in our opinion will lead to further improvement in this topic. First of all, we see great potential in more extensive feature extraction, especially in form of adding sentiment vector alongside lyric embedding vector. This sentiment can be extracted directly from the lyrics themselves, hence does not break the assumption of incorporating only the text of the songs. 

There can be observed a positive correlation between a model's complexity and performance on a test dataset, which leads to the conclusion that the potential of embedding is still not exhausted and a classification layer with a greater number of parameters could lead to better overall performance. We suspect that densely connected layers of neurons could outperform models presented in this work, but this hypothesis has to be put to the test.

Another matter, which could be addressed, is separating a single classification model into two models: the first one for classes with a disproportional number of observations and the second for a subset of data, which can be treated as a balanced classification task. Music genres present in the dataset from this work were mostly shadowed by one music genre, i.e. \textit{Rock}. We see the potential of firstly classifying genre as rock vs non-rock and if a prediction is made for a non-rock class then using the second model for detailed classification.

Lastly and possibly the most obvious, yet still valuable addition would be to collect more data, especially from the last years and from classes which are underrepresented in current datasets. Most aspects mentioned above are to be addressed in our following work. 


\printbibliography

\end{document}


