%
% File main.tex
%
% Contact: car@ir.hit.edu.cn, gdzhou@suda.edu.cn
%%e.agirre@ehu.es or Sergi.Balari@uab.es
%% and that of ACL 08 by Joakim Nivre and Noah Smith

\documentclass[11pt]{article}
\usepackage{comment}
\usepackage{acl2015}
\setlength\titlebox{6cm}
\usepackage{times}
\usepackage{url}
\usepackage{hyperref}
\usepackage{latexsym}
\usepackage{biblatex}
\addbibresource{bibliography.bib}
\usepackage{csquotes}
\usepackage{tabularx}

%\setlength\titlebox{5cm}

% You can expand the title box if you need extra space
% to show all the authors. Please do not make the title box
% smaller than 5cm (the original size); we will check this
% in the camera-ready version and ask you to change it back.


\title{Music genre classification based on song lyrics - comparison between different word embedding techniques and classifiers \\Project Proposal for NLP Course, Winter 2022}

\author{Bartłomiej Eljasiak \\
  Warsaw University of Technology \\
  {\tt\small bartlomiej.eljasiak.stud@pw.edu.pl} \\\And
  Aleksandra Nawrocka \\
  Warsaw University of Technology \\
  {\tt\small aleksandra.nawrocka.stud@pw.edu.pl} \\
  \AND
  Dominika Umiastowska \\
  Warsaw University of Technology \\
  {\tt\small dominika.umiastowska.stud@pw.edu.pl} \\\And 
  supervisor: Anna Wróblewska\\
  Warsaw University of Technology \\
  {\tt\small anna.wroblewska1@pw.edu.pl}\\}

\date{}

\begin{document}
\raggedbottom
% Gdy nie ma ragged bottom to latex dodaje odstępy między paragrafami żeby wypełnić całą przestrzeń na stronie!

\maketitle

\vspace{5em}

\begin{abstract}
Music genre classification (MGC), although a well-known task, still remains challenging in the domain of Music Information Retrieval. We tackle the problem of MGC based solely on lyrics and try to solve it using a solution composed of a state-of-the-art word embedding method tuned for this problem and a separate classification model. Our main contribution is the comparison between different word embedding methods, classification techniques and optimization techniques, which in the domain of MGC is currently lacking. The novelty comes with an additional approach in the form of testing the impact of enriching the words with their sentiment obtained using a separate model.


\begin{comment}
Music genre classification (MGC), although a well-known task, still remains challenging in the domain of Music Information Retrieval. We tackle the problem of MGC based solely on lyrics, but in the process we also test the impact of enriching the words with their sentiment obtained using a separate model. Our solution is composed of a state-of-the-art word embedding method tuned for this problem and a separate classification model. The novelty comes in the form of providing the comparison between different word embedding methods, classification techniques and optimization techniques, which in the domain of MGC is currently lacking. 
\end{comment}
\end{abstract}

% Jakieś linki w podobnej tematyce do naszej pracy
% przejrzałem - Bartek
% https://aclanthology.org/C14-1059.pdf
% https://aclanthology.org/I11-1071.pdf
% https://aclanthology.org/D12-1054.pdf
% https://aclanthology.org/2021.nlp4musa-1.7.pdf
% https://aclanthology.org/2020.nlp4musa-1.6.pdf
% https://archives.ismir.net/ismir2010/paper/000045.pdf
% https://www.researchgate.net/publication/224576802_Exploiting_genre_for_music_emotion_classification


\section{Introduction}
% Skąd się biorą te odstępy????
A music genre is a conventional label on the musical piece which characterizes it as having certain features, conventions, or characteristics. It is quite a complicated problem to say precisely how genres are distinguished. The genre often dictates the style and rhythm of the audio of the song. It seems much harder to define the music genre by lyrics alone, even from a human perspective. Therefore, it is quite an interesting topic to try making such a distinction based on song text. Similar research has already been conducted, but this topic is yet to be fully explored.

The song's lyrics are often related to its melody and rhythm. It is also common for different genres to raise different topics. It was already shown that a combination of audio and text features gets better results than using only audio features \cite{mayer2011Ref}. Furthermore, lyrics may be more accessible and easier to process than audio. Therefore, lyrics classification seems to be an interesting field of study both for its own and for its potential connection with audio features.

In this research, we want to explore different methods for lyrics-based genre classification. Our study will include testing different methods of obtaining text embeddings, such as Continuous Bag-
of-Words, GloVe, word2vec, BERT, and varying classification models, such as Naive Bayes, Support Vector Machine, XGBoost, and Convolutional Neural Network.

We also want to include in our research sentiment analysis of the text. One of the characteristics of the music genre, though rarer considered, is the emotion that the song conveys. We decided to check how exactly those two relate since there seems to be sparse similar research. Therefore, for the above genre classification task, we will additionally consider the emotion detected in the song lyrics and with the use of fusion techniques check how it influences classification performance.


\section{Significance}

Music genre classification (MGC) is at this point a well-known research problem and a subdomain of Music Information Retrieval (MIR). Culture and therefore music avoids strict barriers and definitions, nevertheless, each piece of music is usually categorized into one or more genres. MGC enables us to study this categorization, explore similarities and differences between various genres or even construct a taxonomy. 

In the past, due to heavy computational limitations, the main focus of MGC was put on finding the best features for classification purposes. In \cite{oldFeatures} such features were e.g. \textit{AverageSyllablesPerWord} or \textit{SentenceLengthAverage}. Naturally, word embedding played an important role in extracting information from lyrics and the use of simple methods like \textit{bag-of-words} can be found in various papers \cite{mgc_example_1, liang2011music}. With time, an increasing amount of focus was put strictly on embeddings themselves, developing novel and improved representations. 

Currently, all state-of-the-art approaches for MGC utilizing lyrics rely heavily on word embeddings. In a recent publication \cite{musicWordEmbed} an attempt was made to train the embedding model strictly on lyrics. Unfortunately, the significance of the work is hard to assess due to the lack of usage of this model.

It is also rather common to approach MGC in a multi-modal manner. Usage of the audio itself has to be second if not the most popular source of information with many published articles \cite{audio_1dcnn, audio_attention, audio_reviews_cover, oldFeatures, oldAudio}. Other less trivial data sources are symbolic \cite{symbolic}, culture \cite{oldFeatures}, text reviews \cite{audio_reviews_cover}, and cover art \cite{audio_reviews_cover}. One could say that at this stage researchers experiment with enriching the pieces of music with any meaningful data possible.

To our surprise, we were not able to find any previous research which extracted sentiment from lyrics and used it for purpose of MGC (although a somewhat reversed connection has been studied in \cite{gen2emo}). This is a niche exploration which will be a part of this work. It should be noted that the sentiment of the lyrics will be obtained via model from the lyrics themselves. This means that solution proposed will also base solely on lyrics. 

Reading through the papers approaching MGC in different ways, it is striking that in some cases crucial elements of the proposed solution, are presented without proper justification. In the case of \cite{sig_emb} a 100-dimensional GloVe model was used. It was stated that it is better than another technique called word2vec, but no proof or reference was provided. No other methods were used therefore it is impossible to say what was the value gained from using the GloVe and not e.g. bag-of-words.  In another work \cite{sig_bert} authors used word embeddings obtained from BERT and DistilBERT, with build-on classifiers, then compared their accuracy to BILSTM \cite{bilstm}, which as input received text embedded in an unspecified manner. Numerous simplifications, lack of details and often incomparable results should raise concerns among researchers. How can one declare improvement over some method or even guarantee the value of the proposed work, when provided context for the work is insufficient? Those concerns motivated us to create such context as a result of this project. We want to declare with detail the conditions under which one word embedding method can be described as better for purpose of MGC and test which classification method works best on created lyrics representations. 



\section{Concept and work plan}

Our project is divided into two parts. The main part consists of the comparison between different word embedding techniques, classifiers and optimization methods for MGC. In the second, smaller part we are going to test a modified approach to the above problem.

\subsection{Work plan}

In accordance with overall deadlines, we plan to:
\begin{itemize}
    \item by November 18th have the architecture of the main part of the project prepared and at least one word embedding technique tested,
    \item by November 25th have the main part of the project finished and the architecture of the modified approach prepared,
    \item by December 9th have both parts of the project finished, the presentation and plan for the second project prepared,
    \item by December 16th have the report of the first project prepared.
\end{itemize}

\subsection{Risk analysis}

While planning the project we have identified multiple risks, which are presented in the table \ref{tab:risk}. 

\begin{table}[ht]
    \centering
    \begin{tabularx}{0.5\textwidth}{X|X|X}
        \textbf{Risk} & \textbf{Consequence} & \textbf{Mitigation} \\\hline
        not enough time for performing the project because of other obligations & late submission of the project or not finishing it & reducing the project scope by testing smaller numbers of models/techniques \\\hline
        not big enough computational resources to conduct proper experiments & lower quality of conducted experiments & - \\
    \end{tabularx}
    \caption{Risk analysis}
    \label{tab:risk}
\end{table}

Regarding the second part of the project, as we define it as a hypothesis rather than a thesis, we do not consider unsatisfactory results of the modified approach as a failure. We are interested in the outcome of conducted experiments but we do not have high expectations when it comes to accuracy.



\section{Approach and research methodology}\label{approach}

\subsection{Datasets}\label{datasets}
We have found two datasets which we decided to work on:
\begin{itemize}
    \item \textit{Song lyrics from 79 musical genres} dataset from Kaggle website \cite{KaggleDataset},
    \item \textit{MetroLyrics} dataset processed and put in a GitHub repository \cite{GithubDataset}.
\end{itemize}

In the description of the first dataset, we can find the information that the dataset consists of 379 893 song lyrics from 4239 artists. Around 50\% of the song lyrics are in English and we test our models on them. Information about the artists is kept in a separate file and contains a list of music genres each artist is connected with. As we predict only one music genre for each song we preprocess this dataset by reducing these lists to individual genres (we take the first one from the list) and assigning them to song lyrics of appropriate artists. Furthermore, we preprocess song lyrics as they contain punctuation and span across multiple lines.

By contrast, the second dataset required minimal work on our site. It was initially published on Kaggle website and consisted of 362 237 song lyrics from 18231 artists. The majority of song lyrics (probably around 60\%) were in English. Unfortunately, this dataset was removed from Kaggle website and we were not able to find it in its original form anywhere else. We have found a preprocessed version of it in a GitHub repository of a students' project performed by University of California students in 2018. This version's song lyrics have punctuation removed and contain only one genre for each entry.

\subsection{Datasets preparation}\label{Datasets preparation}

First of all, we conducted some basic preparations of datasets.
For \textit{MetroLyrics} we decided to remove genre \textit{Other} and merge genres \textit{Country} and \textit{Folk} since  our research showed they are very similar, and additionally the second was significantly smaller in samples. As for \textit{Song lyrics from 79 musical genres}, we filtered songs to only English ones and omitted the genre \textit{Pop/Rock} since it is ambiguous.

After conducting a few simple tests on the whole dataset we decided to limit ourselves to a much smaller part of the observations. There were two main reasons for that: limited resources and time - we would not manage to conduct all desired tests on the whole dataset, and second - the dataset was very strongly unbalanced and therefore accuracy was often significantly bigger than balanced accuracy. To solve both problems we decided to limit ourselves to only five genres with the biggest number of samples: \textit{Rock}, \textit{Pop}, \textit{Metal}, \textit{Hip-hop} and \textit{Country}.

In the end, we conducted tests on two datasets:
\begin{itemize}
    \item Balanced dataset with lyrics from 5 genres, $116,120$ observations in total, created from both \textit{MetroLyrics} and \textit{Song lyrics from 79 musical genres} datasets.

    \item Unbalanced dataset with lyrics from 5 genres, $262,122$ observations in total, created from both \textit{MetroLyrics} and \textit{Song lyrics from 79 musical genres} datasets.
\end{itemize}

\subsection{Text preprocessing}

Since in our project, we use embeddings that may be or even should be used on the raw text we decided to test two approaches: one with strong and the second with weak preprocessing of text.

Weak preprocessing consists of deleting numbers in the text and some words characteristic of lyrics notation (e.g. "VERSE", "CHORUS", "2x"). We decided to also expand contractions since they were proven to be troublesome in further preprocessing such as tokenization or removing stopwords. We deleted all special characters since interpunction was used inconsistently in most songs (e.g. lyrics usually didn't have a division for sentences). We also lowered the whole text since every verse began with a capital letter which usually had nothing to do with starting a sentence. Thanks to that we could also use uncased versions of embeddings.

Strong preprocessing consists of, besides the above, tokenization, lemmatization, and deleting stop words. For these steps, we used \textit{nltk} package.



\subsection{Division into two models}

Since one of the datasets is very unbalanced (the amount of Rock genre observations is two times more numerous than all other genres together) we decided to try an approach that will classify the largest genre class separately from the rest. Therefore we prepared a 2-step CNN classifier that consists of two sub-classifiers: a BinaryCNN classifier which only recognizes if lyrics are of the one selected genre or not, and a multiclass CNN classifier (the same is used as a normal classifier for all genres) which classifies all other genres.

\begin{figure}[!h]
\centering
\includegraphics[width=7cm]{plots/2step_cnn_schema.drawio.png}
\caption{2-step CNN architecture}
\end{figure}

In this way, we hope to better recognize and separate the Rock genre and influence significantly balanced accuracy.


\subsection{Creation of the dataset}
We decided to create our own dataset using Spotify API \cite{Spotify} and Genius API \cite{Genius}. We wanted to use Spotify to get tracks from chosen genres and Genius to get lyrics of returned songs. At first, we utilized the \textit{Get Recommendations} endpoint. Unfortunately, the recommended songs repeated a lot between different API calls. That meant that after dropping duplicates we were left with a few times less observations than desired. To deal with this issue, we decided to use the \textit{Search for Item} endpoint. This endpoint allows specifying genre but limits the possible output to the first 1000 tracks. Another problem that we faced was that Genius did not always have the lyrics to the songs provided by Spotify. And even if it did sometimes instead of proper lyrics text of books or different lists of artists and the titles of their songs were returned. Because of that the created dataset needed also manual data cleaning. At last, in that way, we created the dataset containing 5 genres and 4,092 observations. Table \ref{tab:our_dataset_genres} presents the number of observations in each genre.

\begin{table}[h]
\centering
\begin{tabular}{l|r}
\textbf{Genre} & \textbf{Number of observations} \\\hline
country & 896 \\
metal & 887 \\
pop & 815 \\
rock & 767 \\
hip-hop & 727 \\
\end{tabular}
\caption{The number of observations in each genre in created dataset}
\label{tab:our_dataset_genres}
\end{table}


% include your bib file like this:
\printbibliography

\begin{comment}

Bartek:
-> ustalić metodę tworzenia (albo wiele) embeddingów: 
- na zdaniach: BERT...
- na słowach: word2vec...
-> dokumenty na których chcemy bazować - dotyczące klasyfikacji muzyki i ew. coś z sentymentem i dotyczące poszczególnych metod
-> Abstract

Ola:
-> fusion methods (jak połączyć tekst z sentymentem)
-> znaleźć bazy danych (czy 1 czy chcemy połączyć wiele/testować na wielu)
-> Concept and work plan

Dominika:
-> poszukać informacji na temat sentymentu:
- czy chcemy zrobić sentyment na słowach, na zdaniach, na zwrotkach(?) czy na całym tekście.
- ile typów sentymentu chcemy klasyfikować i jakie są dostępne przetrenowane modele (ew. poszukać zbiorów danych z tekstami piosenek i sentymentem)
-> przetestujemy kilka standardowych metod optymizacji + harris hawk.
-> architektura klasyfikatora (sieć neuronowa lub XGBoost, SVM...) <- też możemy potestować różne.
-> The scientific goal of the project

Na później:
-> Ustalić całą architekturę rozwiązania, przepływ danych...


\end{comment}


\end{document}


