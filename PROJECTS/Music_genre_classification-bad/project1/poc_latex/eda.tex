\section{Data preprocessing}
In order to prepare our data for usage we had to process it appropriately. This was done in a few steps. The result of the final step needed to be a text which can be directly fed into the embedding algorithm that produces the embeddings. First, we filtered out the non-english lyrics, since the encoders we are going to use were trained on this specific language. In the following step, we removed artefacts and unimportant words from our lyrics, such as commas, punctuation marks or words like \textit{verse} or \textit{chorus}, as they typically were not part of the lyrics, but represented the structure of the songs. In the last step, we removed infrequent classes from one of the datasets. This action did not change the fact that our problem is significantly unbalanced since music genres' counts differ hugely from one another in both datasets. 

